% !TeX spellcheck = en_GB

\section{Problem 1}

We mentioned in class that citation networks (networks representing
citations between research articles) are, in theory, directed and
acyclic graphs (DAGs).
\begin{enumerate}
	\item Explain why we expect citation networks to be directed and acyclic.
	\item However, in practice, we expect that citation networks have a small
	number of cycles. Why?
	\item Assume that we want to test whether a given citation network has
	cycles. Propose an algorithm to detect if a directed graph has a
	cycle. What is the running time of your algorithm?
\end{enumerate}


\subsection{Expected properties}

A citation network, as said, is a network representing citations between
research articles: research articles are represented through nodes, and
a citation from an article $A$ to another
article $B$ is represented as a directed edge from
$A$ to $B$, i.e. $\left(A,B\right) \in E$. Notice that
edges have to be \textbf{directed} since citations are not symmetric,
i.e. not necessarily if $A$ cites $B$, then
$B$ cites $A$.\\
If we imagine each article having a unique time-stamp (i.e. each article
coming into existence at a unique point in time, thus establishing
a \textbf{total order} of the articles), we can only have an edge
$\left(B, A\right)$ if $A$ precedes $B$ in time,
i.e. $\left(B,A\right) \in E \ \Rightarrow \ A<B$. In other words each article can cite only
articles that precede it. Hence, it's not possible for the graph to have a loop, cause this would
mean that some article is actually citing an article that follows it.\\
More formally: suppose we have a loop $A \leftarrow B \leftarrow \dots \leftarrow A$, then this would mean that $A < B < \dots < A$, clearly an absurdity.

\subsection{Real citation networks}

In practice articles don't really \textit{"come into existence at a unique point in time"}, cause they are edited, reviewed, updated, etc.
In other words, in real life we can't really establish a total order of
the articles (at most a \textit{partial order} if we prevent modifications
after publication). Thus in practice we end up with a small number of
cycles. For example it is not uncommon for two
related, \textit{concurrent} articles to cite each other, thus creating an
obvious loop.

\subsection{Loop checking}

A simple DFS (Depth First Search) does the trick. If we find a
\textit{back edge} during the traversal (i.e. an edge from a node to
itself, or to one of its ancestors in the tree produced by DFS), then
the graph is cyclic, otherwise it isn't. What follows is the pseudocode 
of the algorithm to perform this check.
\newpage
\begin{algorithm}
	\caption{Detect presence of cycles in citation network}
	\begin{algorithmic}[1]
		\ForAll{$v \in V$}
			\State mark $v$ as $UNVISITED$
		\EndFor
		\State $S \ \leftarrow \ \text{empty stack}$
		\State pick any $u \in V$, mark $u$ as $PENDING$
		\State $S.push(u)$
		\While{$S$ is not empty}
			\State $v \ \leftarrow \ S.peek()$
			\State $has\text{-}unvisited\text{-}neighbors \ \leftarrow \ false$
			\ForAll{$(v,z) \in E$}
				\If{$z$ marked as $PENDING$}
					\State \Return TRUE
				\ElsIf{$z$ marked as $UNVISITED$}
					\State $has\text{-}unvisited\text{-}neighbors \ \leftarrow \ true$
					\State mark $z$ as $PENDING$
					\State $S.push(z)$
				\EndIf
			\EndFor
			\If{$\neg \ has\text{-}unvisited\text{-}neighbors$}
				\State mark $v$ as $CLOSED$
				\State $S.pop()$
			\EndIf
		\EndWhile
		\State \Return FALSE
	\end{algorithmic}
\end{algorithm}
\medskip

\noindent Time complexity, as for DFS, is $\mathcal{O}(|V|+|E|)$.
