% !TeX spellcheck = en_GB

\section{Problem 3}

Describe an algorithm to compute exactly the diameter of a given graph. What is the running time of your algorithm?\\
Describe an algorithm that uses the Flajolet-Martin sketching scheme to compute approximately the diameter of a given graph.
What is the running time of your algorithm? How can you control the trade-off between accuracy and speed?

\subsection{Exact solution}

Let $G$ be a graph. Let $V$ be the set of vertices of $G$, such that $|V| = n$, and $E$ be the set of edges of G, such
that $|E| = m$. We define the \textbf{diameter} of $G$ as
\begin{align*}
D(G) = \max_{u,v \in V} \ d(u,v)
\end{align*}
i.e. the greatest \textbf{shortest path} (distance) between any pair of vertices in $G$.\\
We can derive an algorithm to compute the \textbf{exact} diameter of $G$ by trivially applying the definition, that is:
\begin{enumerate}
	\item Compute the shortest path between each pair of vertices of $G$. 
	\item Find the greatest length of any of these paths, that is $D(G)$.
\end{enumerate}
The most efficient way to implement the first step is Dijkstra's algorithm\cite{dijkstra}. Indeed, it has a very nice
property: computing the shortest paths from a vertex $u$ to each of the other vertices of $G$ costs the same of computing
the shortest paths from $u$ to a specific vertex $v$. Note that we need to perform \cite{dijkstra} starting from each
$u \in V$ to get all the shortest paths (needed for step 2). Hence, the time complexity of the whole algorithm is
\begin{align*}
\mathcal{O}(n \cdot m\log{n})
\end{align*}

\subsection{Approximate solution}

Since the algorithm to compute the exact diameter is costly with respect to big graphs, we are interested in finding a good \textbf{approximation}.\\
In what follows, we propose a solution that is based on "\textbf{diffusion}" approach, used in practice to compute the distance distribution of a given graph. Let $B(v,t)$ be the set of vertices within distance $t$ from $v$, i.e.
\begin{align*}
B(v,t) = \{ z \in V \ | \ d(v,z) \le t \} \qquad v \in V, \ t \ge 0
\end{align*}
In particular, we note that
\begin{align}
B(v,D(G)) = V \quad \forall v \in V \label{ball_diameter}
\end{align}
The key idea of "diffusion" is that this set can be defined inductively as follow:
\begin{align}
&B(v,0) = \{ v \} \nonumber\\
&B(v,t+1) = \bigcup_{(v,z) \in E} B(z,t) \ \cup \ \{ v \} \label{ball}
\end{align}
The intuition underlying (\ref{ball}) is that vertices at distance $t+1$ from the current location can be reached in $t$ hops from each one of the neighbours. Even though it is trivial to design an algorithm that implements the inductive definition, keeping all the sets in memory and performing union among them is costly (indeed, space complexity is $\mathcal{O}(n^2)$).\\
Therefore, we need to apply \textbf{Flajolet-Martin sketching}\cite{fm} scheme, that is an algorithm for approximating the number of distinct elements in a stream (or set, in this case) with a single pass and space complexity which is \textbf{logarithmic} in the maximum number of possible distinct elements. In this settings, "diffusion" sets are implemented as \textbf{bitmasks} of $\log{n}$ bits each and union is computed efficiently through \textbf{bitwise OR} operation. What follows is the algorithm we designed.
\medskip
\begin{algorithm}
\caption{Approximate diameter computation}
\begin{algorithmic}[1]
	\ForAll{$v \in V$}
		\State $B(v,0) = log{n}$ bits bitmask s.t. only 1 bit is set and bit $i$ has probability \State $\frac{1}{2^{i+1}}$ to be set
	\EndFor
	\ForAll{distance $t$ \textbf{from} 1}
		\ForAll{$v \in V$}
			\State $B_{last}(v), \ B(v,t) = B(v,t-1)$
		\EndFor
		\ForAll{$(u,v) \in E$}
			\State $B(u,t) = B(u,t) \quad BITWISE-OR \quad B_{last}(v)$
		\EndFor
		\ForAll{$v \in V$}
			\State $R =$ position of leftmost non-zero bit in $B(v,t)$
			\State $IS(v,t) = 2^R$
		\EndFor
		\State $S(t) = \dfrac{1}{|V|} \cdot \sum_{v \in V}IS(v,t)$
		\If{$|V| - \epsilon < S(t)$} 
			\State \Return t 
		\EndIf
	\EndFor
\end{algorithmic}
\end{algorithm}

\medskip
\noindent Note that we introduced value $\epsilon$ (line 17), standing for the \textbf{error} we accept to tolerate when estimating $D(G)$. Indeed, the computation stops when the average cardinality of the "diffusion" sets is \textbf{close} to $|V|$, i.e. differs from it by at most $\epsilon$. Hence, because of (\ref{ball_diameter}), we can conclude that the last value of $t$ is close to the exact value of $D(G)$.\\
The algorithm includes at least a linear scanning of vertices and edges for each iteration and performs a number of iterations equal to $D(G)$. Let $d$ be such value. Hence, the time complexity of the whole algorithm is
\begin{align*}
\mathcal{O}(d \cdot (n + m))
\end{align*}
Since we necessarily need to iterate over vertices and edges to compute "diffusion" sets, the only way to reduce the complexity is to manipulate $d$, thus, the number of times that the main loop is executed. To reduce its value, we can only increase the value of $\epsilon$, losing accuracy. Therefore, $\epsilon$ is the parameter through which we can \textbf{tune} the trade-off between accuracy and speed.









