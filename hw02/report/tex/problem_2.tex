% !TeX spellcheck = en_GB

\section{Problem 2}

Assume that a graph is stored in the edge incidence model, that is, all edges incident to one vertex are stored sequentially.\\
Our sampling algorithm for estimating the triangle coefficient relies on sampling one connected triple of the graph with uniform probability. Namely, all connected triples should have equal probability to be sampled. Describe a three-pass algorithm to sample a connected triple uniformly at random.\\
\textbf{Note:} This is already outlined in the slides, you are asked to describe the method in more detail.

\subsection{Connected triple sampling}

Let $G$ be a graph. Let $V$ be the set of vertices and $E$ be the set of edges of $G$. We define $\langle u,v,z\rangle$ as a \textbf{connected triple} centred in $v$, i.e. a triple of vertices of $G$ such that $(u,v) \in E$ and $(v,z) \in E$ (where edges are assumed undirected). Hence, the total number of connected triples in $G$ is
\begin{align*}
CT(G) = \sum_{v \in V} CT(G,v) = \ \sum_{v \in V} \binom{d_v}{2} \ = \ \sum_{v \in V} \dfrac{d_v(d_v - 1)}{2}
\end{align*}
Note that $d_v$ stands for the \textbf{degree} of vertex $v \in V$.\\
Since we use \textbf{incidence model} to store the edges, if $G$ is \textbf{undirected}, then
\begin{align*}
(u,v) \in E \ \Rightarrow \ (v,u) \in E
\end{align*}
Note that the sub-list of edges that are incident in $v \in V$ has length equal to $d_v$. It is trivial to deduce that each element adds a number of connected triples centred in $v$ equal to its \textbf{0-based position} in this sub-list. For instance, suppose we have following incidence sub-list for $a \in V$:
\begin{align*}
	0. \quad &a,b\\
	1. \quad &a,c\\
	2. \quad &a,d\\
	\ldots
\end{align*}
Following picture shows how the graph grows as we scan the list:\\
\begin{figure}[!htbp]
	\begin{subfigure}[b]{0.32\textwidth}
		\centering
		\begin{tikzpicture}[>=stealth, every node/.style={circle, draw, minimum size=0.75cm}]
		\graph [tree layout, grow=down, fresh nodes, level distance=0.5in, sibling distance=0.5in]
		{ a <-> {b} };
		\end{tikzpicture}
		\caption*{$CT(G,a) = 0$}
		\label{fig:CT0}
	\end{subfigure}
	\begin{subfigure}[b]{0.32\textwidth}
		\centering
		\begin{tikzpicture}[>=stealth, every node/.style={circle, draw, minimum size=0.75cm}]
		\graph [tree layout, grow=down, fresh nodes, level distance=0.5in, sibling distance=0.5in]
		{ a <-> {b, c} };
		\end{tikzpicture}
		\caption*{$CT(G,a) = 1$}
		\label{fig:CT1}
	\end{subfigure}
	\begin{subfigure}[b]{0.32\textwidth}
		\centering
		\begin{tikzpicture}[>=stealth, every node/.style={circle, draw, minimum size=0.75cm}]
		\graph [tree layout, grow=down, fresh nodes, level distance=0.5in, sibling distance=0.5in]
		{ a <-> {b, c, d} };
		\end{tikzpicture}
		\caption*{$CT(G,a) = 3$}
		\label{fig:CT3}
	\end{subfigure}
\end{figure}

\noindent As we can see, when $(a,d)$ is read, two new triples are created: $\langle b,a,d\rangle$ and $\langle c,a,d\rangle$. These ones have to be added to triples created in previous steps, that is the only $\langle b,a,c\rangle$, in this case. Indeed, the number of connected triples centred in a vertex $v$ is equal to the \textbf{number of possible pairs involving its neighbours}, that is $\binom{d_v}{2}$.\\
What follow are the \textbf{two passes} of the algorithm we designed to sample a connected triple uniformly at random, based on previous considerations.
\medskip
\begin{algorithm}
	\caption{$1^{st}$ pass: compute $CT(G)$}
	\begin{algorithmic}[1]
		\State $v \ \leftarrow \ edges[0][0]$ \Comment pick the first vertex in the first edge of the stream
		\State $v_{iedix} \ \leftarrow \ 0$ \Comment 0-based index in incidence sub-list of v
		\State $total \ \leftarrow \ 0$
		\ForAll{$e \in edges \setminus \{v\}$}
			\If{$e[0] = v$}
				\State $v_{iedix} \ \leftarrow \ v_{iedix} + 1$
				\State $total \ \leftarrow \ total + v_{iedix}$
			\Else \Comment reset
				\State $v \ \leftarrow \ e[0]$
				\State $v_{iedix} \ \leftarrow \ 0$
			\EndIf
		\EndFor
		\State \Return $total$
	\end{algorithmic}
\end{algorithm}

\newpage
\begin{algorithm}
	\caption{$2^{nd}$ pass: sample connected triple uniformly at random}
	\begin{algorithmic}[1]
		\State $i \ \leftarrow$ random integer in $[1, CT(G)]$
		\State $v \ \leftarrow \ edges[0][0]$
		\State $v_{iedix} \ \leftarrow \ 0$
		\State $v_{edges} \ \leftarrow \ [edges[0]]$ \Comment running list of edges incident to v
		\State $total \ \leftarrow \ 0$
		\ForAll{$e \in edges \setminus \{v\}$}
			\If{$e[0] = v$}
				\State $v_{edges}.append(e)$
				\State $v_{iedix} \ \leftarrow \ v_{iedix} + 1$
				\State $total \ \leftarrow \ total + v_{iedix}$
				\If{$total \ge 1$} \Comment build and return $i\text{-}th$ triple (deterministic)
					\State $u \ \leftarrow \ e[1]$ \Comment picked from first edge s.t. $total \ge 1$ holds
					\State $w \ \leftarrow \ v_{edges}[total-i][1]$ \Comment picked from $(total-i)\text{-}th$ incident edge
					\State \Return $\langle u,v,w \rangle$
				\EndIf
			\Else \Comment reset
				\State $v \ \leftarrow \ e[0]$
				\State $v_{iedix} \ \leftarrow \ 0$
				\State $v_{edges} \ \leftarrow \ [e]$
			\EndIf
		\EndFor
	\end{algorithmic}
\end{algorithm}




