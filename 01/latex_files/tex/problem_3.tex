% !TeX spellcheck = en_GB

\section{Problem 3}

A group of n man and m women go to a Chinese restaurant and sit in a round table,
such that each person has to other person next to him/her.
\begin{itemize}
	\item[1.] Describe a sample space that describes the random process.
	\item[2.] Find the expected number of men who will be sitted next to at least one woman.
\end{itemize} 


\subsection{Probability space definition}

Let $\Omega$ be the probability space of this process. It consists in all possible permutations of the people sitting around the table (assuming, without loss of generality, an order among the seats). In addition, we have that $\Omega$ is an equiprobable space, since every permutation of the people has the same probability to be chosen (at random). Therefore,
\begin{align*}
	|\Omega| = (n+m)!
\end{align*}
\begin{align*}
	\Pr(\omega) = \dfrac{1}{|\Omega|} = \dfrac{1}{(n+m)!} \qquad \forall \ \omega\in\Omega
\end{align*}


\subsection{Expectation}

Let $X$ be a discrete random variable representing the total number of men sitted next to at least a woman, defined as
\begin{align*}
	X: \Omega \rightarrow \{ x \in \mathbb{N} \ | \ 0 \leq x \leq n \}
\end{align*}
Let $X_i$ be a Bernoulli random variable representing whether the $i \textendash th$ man is sitted next to at least a woman or not. Hence,
\begin{align*}
	X_i = 
	\left\{ \begin{aligned}
		1 &\quad\text{if man is sitted next to at least a woman}\\
		0 &\quad\text{otherwise}
	\end{aligned}\right.
\end{align*}
Therefore, we have that
\begin{align}
	X = \sum_{i = 1}^{n}X_i \quad\Rightarrow\quad E\left[ X \right] = \sum_{i = 1}^{n} E\left[ X_i \right] \label{lin_rv}
\end{align}
Note that we can write (\ref{lin_rv}) because of the linearity of expectation.\\
Therefore, we need to compute $E\left[ X_i \right]$ in order to get the desired expected value.
\begin{align}
	E\left[ X_i \right] &= \sum_{k \in \mathrm{range}(X_i)} k \cdot \Pr(X_i = k)\nonumber\\
	&= \Pr(X_i = 1)\nonumber\\
	&= 1 - \Pr(X_i = 0)\nonumber\\
	&= 1 - \dfrac{(n+m) \cdot \binom{n-1}{2} \cdot 2 \cdot (n+m-3)!}{(n+m)!} \label{pr_xi0}\\
	&= 1 - \dfrac{(n-1) \cdot (n-2)}{(n+m-1) \cdot (n+m-2)} \label{e_xi}
\end{align}
Let us focus on result (\ref{pr_xi0}). We trivially computed $\Pr(X_i = 0)$, i.e. the probability that a man is not seated next to a woman, by enumerating all cases in which the $i \textendash th$ man is sitted next to other two men. Indeed, there are:
\begin{itemize}
	\item $n+m$ possible seats for the $i \textendash th$ man.
	\item $\binom{n-1}{2}$ ways to choose other two men.
	\item 2 ways to arrange other two men ("around" the $i \textendash th$ man).
	\item $(n+m-3)!$ permutations of the remaining people.
\end{itemize}
Hence, substituting (\ref{e_xi}) into (\ref{lin_rv}), we have that
\begin{align}
	E\left[ X \right] = \left( 1 - \dfrac{(n-1) \cdot (n-2)}{(n+m-1) \cdot (n+m-2)} \right) \cdot n
\end{align}














