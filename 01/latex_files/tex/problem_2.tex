% !TeX spellcheck = en_GB

\section{Problem 2}

You throw a set of 3 regular dice again and again, until for the first time you see a
sum of 11 or a sum of 16.
\begin{itemize}
	\item[1.] Design an appropriate probability space for the above process.
	\item[2.] What is the probability that you stop because you see a sum of 16?
\end{itemize}

\subsection{Probability space definition}

Let $D = \{1, \ldots, 6\}$ be the set of all values that can be a result of a die throw. Let $\Omega$ be the probability space of this process. Therefore,
\begin{align*}
	\Omega = D \times D \times D
\end{align*}
since it is composed by all possible combinations of the values of 3 dice. In addition, $\Omega$ is an equiprobable space. Indeed,
\begin{align*}
	\Pr(\omega) = \dfrac{1}{|\Omega|} = \dfrac{1}{6^3} = \dfrac{1}{216} \qquad \forall \ \omega\in\Omega
\end{align*}

\subsection{Event probability}

Let $X$ be a discrete random variable representing the sum of the values of 3 dice, defined as
\begin{align*}
	X: \Omega \rightarrow \{ x \in \mathbb{N} \ | \ 3 \leq x \leq 18 \}
\end{align*}
and such that
\begin{align}
	\Pr(X = x) = \dfrac{|\{ \langle y,w,z \rangle \ | \ y,w,z \in D \ \wedge \ y+w+z = x \}|}{|\Omega|} \label{triplet}
\end{align}
Let $E$, $E_i \in \Omega$ be two events such that
\begin{align*}
	E &= \text{"stop because you get a sum of 16"}\\
	E_i &= \text{"you get a sum of 16 only at $(i+1) \textendash th$ throw"}
\end{align*}
Therefore,
\begin{align}
	\Pr(E) &= \Pr\left(\bigcup_{i = 0}^{\infty} E_i \right) \nonumber\\
		&= \sum_{i = 0}^{\infty}\Pr(E_i) \label{disj2.1}
\end{align}
Note that we can write (\ref{disj2.1}) since the events are disjoint each other.\\
Therefore, we only need to compute $\Pr(E_i)$ in order to get the desired result. Since, by definition of the problem, every throw is independent from all the others
\begin{align}
	\Pr(E_i) &= \Pr( X \ne 11 \ \cap \ X \ne 16 )^i \cdot \Pr(X = 16) \nonumber\\
			&= [1 - \Pr( X = 11 \ \cup \ X = 16)]^i \cdot \Pr(X = 16) \nonumber\\
			&= [1 - (\Pr(X = 11) + \Pr( X = 16))]^i \cdot \Pr(X = 16) \label{disj2.2}
\end{align}
Note that we can write (\ref{disj2.2}) since $X = 11$ and $X = 16$ are two disjoint events, because it is not possible to have a triplet (as we defined it in (\ref{triplet})) whose sum is both 11 and 16.\\
Therefore, we need to compute $\Pr(X = 11)$ and $\Pr( X = 16)$ in order to get $\Pr(E_i)$. We can trivially do it by enumerating all the triplets whose sum is 11 and 16, respectively, and then dividing by $|\Omega|$. Hence,

\begin{itemize}

	\item $X = 11$
	\begin{align*}
		\underbrace{
			\begin{aligned}
				\langle 1,4,6 \rangle\\
				\langle 2,3,6 \rangle\\
				\langle 2,4,5 \rangle
			\end{aligned}
		}_{\text{3! permutations each}}
		&&\underbrace{
			\begin{aligned}
				\langle 1,5,5 \rangle\\
				\langle 3,3,5 \rangle\\
				\langle 3,4,4 \rangle
			\end{aligned}
		}_{\text{3 permutations each}}\\
	\end{align*}
	\begin{align*}
		\Pr(X = 11) &= \dfrac{3 \cdot 3! + 3 \cdot 3}{216} = \dfrac{1}{8}
	\end{align*}
	
	\item $X = 16$
	\begin{align*}
		\underbrace{
			\begin{aligned}
				\langle 4,6,6 \rangle\\
				\langle 5,5,6 \rangle
			\end{aligned}
		}_{\text{3 permutations each}}
		&&\Pr(X = 16) = \dfrac{2 \cdot 3}{216} = \dfrac{1}{36}
	\end{align*}

\end{itemize}
Therefore, from (\ref{disj2.2}), we can conclude that
\begin{align}
	\Pr(E_i) &= \left( 1 - \dfrac{1}{8} - \dfrac{1}{36} \right)^i \cdot \dfrac{1}{36} = \left( \dfrac{61}{72} \right)^i \cdot \dfrac{1}{36} \label{pr_ei}
\end{align}
and, because of (\ref{pr_ei}), that
\begin{align}
	\Pr(E) = \dfrac{1}{36} \cdot \sum_{i = 0}^{\infty}\left( \dfrac{61}{72} \right)^i = \dfrac{1}{36} \cdot \dfrac{1}{1 - \dfrac{61}{72}} \simeq 0.18 \label{pr_e}
\end{align}
Note that we can write (\ref{pr_e}) since we obtained a convergent geometric series.\\
